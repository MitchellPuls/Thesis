This research is conducted as part of the final thesis of Mitchell Quinn Puls, Technical Computing student at the Amsterdam University of Applied Sciences. This research is about the effects of a high amount of computers, processing a vast amount of data. This research is, in assignment from CERN in Switzerland. 
\section{CERN}
CERN is a European organization for  nuclear research, situated in Geneve Switzerland. CERN was founded in 1954 and is one of Europe's first joint ventures. The main goal is to study the fundamental structure of the universe, by researching matter and particles using purpose built particle accelerators and detectors. Particle accelerators beam particles to high energies, before colliding them against each other against or a stationary object. Detectors record and observe this collision \footnote{https://home.cern/about}. One of these detectors is ALICE
\section{ALICE}
ALICE stands for A Large Ion Collider Experiment, and is a detector mounted on the Large Hadron Collider at CERN.  ALICE's main function is to study matter at extreme energy densities, where matter turns into a form called quark-gluon plasma. Everything in the universe is made from protons, neutrons (except hydrogen which does not have any neutrons) and electrons. Protons and neutrons are then build up with quarks and bound together with something called a gluon. Quark-gluon plasma is matter that appeared at the very start of the big bang, and is the matter that appears when the quarks and gluons are seperated from each other. CERN wants to observe this matter. The way they achieve this, is by shooting two lead ions against each other. This produces heat that is over 100,000 times hotter than the center of the Sun. This breaks the bounds between the quarks and the gluons and makes the quark-gluon plasma visible \footnote{https://home.cern/about/experiments/alice}.
\subsection{Upgrade}
In July 2018 the accelerator will be stopped for around 18 months for a planned upgrade of the ALICE detector. (van der Lee, 2017, p. 1) During this period, CERN is upgrading it's hardware and software. This upgrade is in collaboration with various schools and universities throughout Europe, including the Amsterdam University of Applied Science. One of these upgrades is an algorithm for Load Balancing. In 2020, ALICE will restart with it's new upgraded detector. ("Technical Design Report for the Upgrade of the Online Offline Computing System", 2015, p. i)

\section{Load Balancing}
The data stream that comes from ALICE is equal to about 1.1 Terabyte per second. All of this data comes in what is known as a heartbeat. This heartbeat gets distributed over 250 First Level Processors and funneled through	1500 Event Processing Nodes. The efficient distribution of this process, and also the handling of data in case of a failure in the system, is what is known as Load Balancing. All of these computers are monitored by an Information Node.