\section{Conclusion}
The new setup is a viable way of conducting experiments for testing load balancing algorithms. With a Pearson correlation value of 0.95 and higher for the three experiments, we can conclude that there is a very strong positive linear correlation. Further more looking at the data for the expansion of the experiment, we can see that the Information Node does not have a negative effect with handling the extra units. In fact, because it has more time to handle the units in between round robins, less TFs are lost in the process. \\
Lastly, there is a difference in the amount of TFs lost depending on the layout of the system. With each cluster fail-over With the Blacklist algorithm there will be a higher TF loss if the IPs of the machines that have a fail-over are more spread out, as compared to the IPs being one after another. 

\section{Recommendations}
\subsection{Do the experiment with Ansible induced fail-overs}
Both these experiments are done with fail-overs that were induced using the TF that was send at that point. This means that the fail-overs are in a sense predictable, and means that there will always be atleast one TF lost. It is recommended to create scripts that induce a fail-over using Ansible at random time intervals. This to prevent having the one certain TF loss, and to randomize the moment when an EPN will have a fail-over.
\subsection{Use a lowest latency approach, as opposed to a round robin for the distribution}
Currently the distribution is done using a Round Robin approach. This means that the distribution will go from the top of the list, to the bottom each time. This means that if there is a fail-over induced by a TF, it will have the entire list still as a buffer to be targeted again. If there is a lowest latency approach used, then the Information Node will target the EPN that is best suited at the time to receive the TF. If there is a fail-over happening at some point, then this EPN won't be targeted anymore by default.